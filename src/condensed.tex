\subsection{Condensed abelian groups}

\begin{remark}
  For the time being, the following facts will be used without proof in this text.
  (They have or will be formalized in Lean though.)

  \begin{itemize}
    \item There is a natural functor $\Top \to \Cond(\Sets)$.
    \item The category of condensed abelian groups (resp.~condensed $R$-modules)
      is an abelian category with enough projectives.
      For $S$ an extremally disconnected set, the objects $\mathbb Z[S]$ (resp.~$R[S]$) is projective.
    \item We write $H^i(S, M)$ for $\text{Ext}^i(\mathbb Z[S], M)$.
  \end{itemize}
\end{remark}

\begin{definition}
  \label{exact-with-constant}
  \lean{CompHausFiltPseuNormGrp₁.exact_with_constant}
  \leanok
  \uses{png-cats}
  Consider an exact sequence of abelian groups
  \[ X'\xrightarrow{f} X\xrightarrow{g} X'' \]
  such that all of $X'$, $X$ and $X''$ carry the structure of compact-Hausdorffly-filtered-pseudonormed abelian groups.
  Assume that the maps are strict, i.e., $f(X'_{\leq c})\subset X_{\leq c}$ and $g(X_{\leq c})\subset X''_{\leq c}$.
  We say that the sequence is \emph{exact with constant $c_f$} if
  $\mathrm{ker}(g)\cap X_{\leq c}\subset f(X'_{\leq c_fc})$.
\end{definition}

\begin{proposition}
  \label{exact-with-constants-lim}
  \lean{CompHausFiltPseuNormGrp₁.exact_with_constant.extend}
  \leanok
  \uses{bounded-limits, exact-with-constant}
  Consider an inverse system
  \[ (X'_i\xrightarrow{f_i} X_i\xrightarrow{g_i} X_i'')_i \]
  of exact sequences that are exact with constant $c_f$ (independent of $i$).
  Moreover, assume that the transition maps $X'_i\to X'_j$, $X_i\to X_j$ and $X''_i\to X''_j$ are strict, and let $X'$, $X$ and $X''$ be their limits.
  Then
  \[ X'\xrightarrow{f} X\xrightarrow{g} X'' \]
  is exact with the same constant $c_f$.
\end{proposition}

\begin{proof}
  \leanok
  Pass to cofiltered limits of compact Hausdorff spaces in the statements $\mathrm{ker}(g)\cap X_{\leq c}\subset f(X'_{\leq c_fc})$,
  noting that cofiltered limits of surjections of compact Hausdorff spaces are still surjective (by an application of Tychonoff).
\end{proof}

\begin{definition}
  \label{CHPNG-to-Cond}
  \uses{png-cats}
  \lean{CompHausFiltPseuNormGrp.to_Condensed}
  \leanok
  There is a natural functor
  \begin{align*}
    \text{CHPNG} &\to \Cond(\Ab) \\
    M &\mapsto \underline{M}
  \end{align*}
  where $\underline{M}(S)$ is defined to be collection of functions $f \colon S \to M$ that factor as continuous through $M_c$, for some $c$.
  In symbols:
  \[ M(S) = \{f \colon S \to M \mid \exists c, f(S) \subset M_c \text{ and $f \colon S \to M_c$ is continuous}\}. \]
\end{definition}

\begin{proposition}
  \label{CHPNG-Cond-exact}
  \lean{Condensed.exact_of_exact_with_constant}
  \leanok
  \uses{CHPNG-to-Cond, exact-with-constant}
  Consider an exact sequence of abelian groups
  \[ X'\xrightarrow{f} X\xrightarrow{g} X'' \]
  such that all of $X'$, $X$ and $X''$ carry the structure of compact-Hausdorffly-filtered-pseudonormed abelian groups.
  Assume that $f(X'_{\leq c})\subset X_{\leq c}$ and $g(X_{\leq c})\subset X''_{\leq c}$.
  If the sequence is exact with constant $c_f$,
  then the sequence
  \[ \underline{X'}\to \underline{X}\to \underline{X''} \]
  of condensed abelian groups is exact.
\end{proposition}

\begin{proof}
  \leanok
  TODO: Rewrite this proof. It is written for the case $\bullet \stackrel{g}{\to} \bullet \stackrel{0}{\to} \bullet$.

  We evaluate at $S\in \mathrm{ExtrDisc}$.
  Any map $S\to X''$ factors over some $X''_{\leq c}$,
  and then $g: X_{\leq c_gc}\times_{X''} X''_{\leq c}\to X''_{\leq c}$ is a surjection of compact Hausdorff spaces;
  as $S$ is extremally disconnected, the map $S\to X''_{\leq c}$ can be lifted,
  showing that $g: X(S)\to X''(S)$ is surjective.
  A similar argument shows that the kernel of $g: X(S)\to X''(S)$ is in the image of $f: X'(S)\to X(S)$,
  the latter clearly being injective.
\end{proof}

\begin{lemma}
  \label{free-png}
  \uses{CHPNG-to-Cond}
  \lean{free_pfpng_profinite_iso}
  \leanok
  Let $S = \varprojlim S_i$ be a profinite set.
  Then $\mathbb Z[S]$ is naturally a profinitely filtered pseudo-normed group,
  via $\mathbb Z[S]_{\le c} = \varprojlim \mathbb Z[S_i]_{\le c}$,
  where $\mathbb Z[S_i]_{\le c}$ is the set $\{\sum_{s \in S_i} n_s[s] \mid \sum_s |n_s| \le c\}$.

  There is a natural isomorphism between the free condensed abelian group $\mathbb Z[S]$
  and the colimit $\bigcup_c \mathbb Z[S]_{\le c}$ of condensed sets.
\end{lemma}

\begin{proof}
  \leanok
  For now, see Lemma~2.1 of \cite{Analytic}.
\end{proof}

\begin{proposition}
  \label{acyclic-sheaf}
  \lean{condensed.acyclic_of_exact}
  \leanok
  Let
  \[ M \colon \mathrm{ProFin}^{\mathrm{op}}\to \mathrm{Ab} \]
  be a functor, i.e.~a presheaf of abelian groups on $\mathrm{ProFin}$. Assume that $M$ preserves finite products, and that for any surjective map $f: T\to S$, the complex
  \[ 0\to M(S)\to M(T)\to M(T\times_S T)\to M(T\times_S T\times_S T)\to \ldots \]
  is exact.

  Then $M$ is a condensed abelian group, and for all profinite sets $S$ and $i>0$, one has $H^i(S,M)=0$ for $i>0$.
\end{proposition}

\begin{proof}
  \leanok
  % Proposition~\ref{prop:CondAbexplicitsheafcond} shows that $M$ is a condensed abelian group.
  We prove by induction on $i>0$ that $H^i(S,M)=0$ for all profinite sets $S$, so assume the vanishing of $\mathrm{Ext}^1,\ldots,\mathrm{Ext}^i$ for some $i\geq 0$.
  (This is vacuous for $i=0$.)
  We aim to prove that $H^{i+1}(S,M)=0$ for all profinite sets $S$.
  Pick any profinite set $S$ and a cover $T\to S$ with $T\in \mathrm{ExtrDisc}$.
  We get a long exact sequence of condensed abelian groups
  \[ \ldots\to \mathbb Z[T\times_S T\times_S T]\to \mathbb Z[T\times_S T]\to \mathbb Z[T]\to \mathbb Z[S]\to 0: \]
  Indeed, taken as presheaves on $\mathrm{ExtrDisc}$, this is already true on the level of presheaves,
  where it reduces to the case of surjections of sets in which case one can write down a contracting homotopy.
  (Actually, the similar result is true in any topos, where one has to maybe argue a bit more carefully.)

  The following argument is making explicit something usually seen through a spectral sequence. Define inductively
  \[ K_1=\mathrm{ker}(\mathbb Z[T]\to \mathbb Z[S]), \]
  \[ K_2=\mathrm{ker}(\mathbb Z[T\times_S T]\to \mathbb Z[T]) \]
  etc. One gets exact sequences
  \[ 0\to K_n\to \mathbb Z[T^{n/S}]\to K_{n-1}\to 0 \]
  for $n\geq 2$. From the long exact sequence
  \[ \ldots\to H^i(T,M)\to \mathrm{Ext}^i(K_1,M)\to H^{i+1}(S,M)\to H^{i+1}(T,M)=0 \]
  we see that we have to prove that $\mathrm{Ext}^i(K_1,M)=0$ (if $i>0$, otherwise that $M(T)$ surjects onto $\mathrm{Hom}(K_1,M)$).
  Assuming $i>0$, we can go on, and using the inductive hypothesis applied to the fibre products $T^{\ast/S}$, we inductively see that
  \[ H^{i+1}(S,M)=\mathrm{Ext}^i(K_1,M)=\mathrm{Ext}^{i-1}(K_2,M)=\ldots=\mathrm{Ext}^1(K_i,M) \]
  and eventually that this is the same as the cokernel of
  \[ M(T^{i/S})\to \mathrm{Hom}(K_{i+1},M).  \]
  But there is an exact sequence
  \[ 0\to \mathrm{Hom}(K_{i+1},M)\to M(T^{(i+1)/S})\to \mathrm{Hom}(K_{i+2},M) \]
  and $\mathrm{Hom}(K_{i+2},M)$ injects into $M(T^{(i+2)/S})$. We see that
  \[ \mathrm{Hom}(K_{i+1},M)=\mathrm{ker}(M(T^{(i+1)/S})\to M(T^{(i+2)/S})) \]
  and we need to see that
  \[ M(T^{i/S})\to \mathrm{Hom}(K_{i+1},M)=\mathrm{ker}(M(T^{(i+1)/S})\to M(T^{(i+2)/S})) \]
  is surjective, which is precisely the exactness of the \v{C}ech complex.
\end{proof}

\begin{proposition}
  \label{normed-to-cond}
  \lean{LCC_iso_Cond_of_top_ab}
  \leanok
  \uses{CHPNG-to-Cond}
  Let $(M,\|\cdot\|)$ be a complete normed group, regarded as a topological group.
  Then the corresponding condensed abelian group $\underline{M}$ sends any profinite set $S$
  to the completion of normed group of locally constant maps $S\to M$ (with the supremum norm).
\end{proposition}

\begin{proof}
  \leanok
  This is a standard result. We omit the proof here, but it is formalized in Lean.
\end{proof}

\begin{proposition}
  \label{normed-to-cond-acyclic}
  \uses{CHPNG-to-Cond}
  \lean{free_acyclic}
  \leanok
  Let $(M,\|\cdot\|)$ be a complete normed group, regarded as a topological group.
  Then for any profinite set $S$, one has $H^i(S,\underline{M})=0$ for $i>0$.
\end{proposition}

\begin{proof}
  \uses{acyclic-sheaf, normed-to-cond}
  \leanok
  This follows Proposition~\ref{acyclic-sheaf} and the part of \cite[Proposition 8.19]{Analytic} that is already formalized.
\end{proof}

\begin{lemma}
  \label{Ext-Lbar}
  \uses{Lbar, normed_with_aut}
  \lean{Lbar.is_iso_Tinv_sub}
  \leanok
  Let $0 < r < r' < 1$ be real numbers.
  Let $S$ be a profinite set, and let $V$ be a $r$-normed (Banach?) $\mathbb Z[T^{\pm1}]$-module.
  Then $\Ext_{\CondMod_{\mathbb Z[T^{-1}]}}^i(\Lbar_{r'}(S), V) = 0$ for all $i \ge 0$.
  In other words,
  \[ \Ext_{\mathbb Z}^i(\Lbar_{r'}(S), V) \xrightarrow{[T⁻¹]_L - [T⁻¹]_V} \Ext_{\mathbb Z}^i(\Lbar_{r'}(S), V) \]
  is a bijection for all $i$.
\end{lemma}

\begin{proof}
  \uses{Ext-Lbar-aux, Qprime-prop, normed-to-cond-acyclic}
  \leanok
  With Proposition~\ref{Qprime-prop}, it suffices to prove the following assertion.
  Pick $1>r'>r>0$, a profinite $S$, and some $r$-Banach $\mathbb Z[T^{\pm 1}]$-module $V$ as before.
  Then we want to prove that
  \[ \mathrm{Ext}^i_{\mathbb Z[T^{-1}]}(Q'(\overline{\mathcal M}_{r'}(S)),V)=0 \]
  for all $i\geq 0$.

  At this point, it is profitable to rewrite this again as the bijectivity of
  \[ (T^{-1})_V - (T^{-1})_{\mathcal M}: \mathrm{Ext}^i(Q'(\overline{\mathcal M}_{r'}(S)),V)\to \mathrm{Ext}^i(Q'(\overline{\mathcal M}_{r'}(S)),V). \]
  Now these Ext-groups can be computed! More precisely, recall that $Q'(\overline{\mathcal M}_{r'}(S))$ is a complex of the form
  \[ \ldots\to \mathbb Z[\overline{\mathcal M}_{r'}(S)^2]\to \mathbb Z[\overline{\mathcal M}_{r'}(S)]\to 0.  \]
  Termwise, the Ext-groups turn into cohomology groups
  \[ H^i(\overline{\mathcal M}_{r'}(S)^{2^j},V).  \]
  Unfortunately, $\overline{\mathcal M}_{r'}(S)$ itself is not profinite, so we cannot directly apply Proposition~\ref{normed-to-cond-acyclic}.
  To get around this last cliff, we write $Q'(\overline{\mathcal M}_{r'}(S))$ as a filtered colimit of complexes
  \[ Q'(\overline{\mathcal M}_{r'}(S))_{\leq c}: \ldots \to \mathbb Z[\overline{\mathcal M}_{r'}(S)^2_{\leq \kappa_1 c}]\to \mathbb Z[\overline{\mathcal M}_{r'}(S)_{\leq \kappa_0 c}]\to 0 \]
  where the constants $\kappa_0=1,\kappa_1,\ldots$ are positive and chosen so that all differentials are well-defined.
  (The possibility of choosing such constants has already been formalized; TODO include pointer.)
  It suffices to prove that
  \[ (T^{-1})_V - (T^{-1})_{\mathcal M}: \mathrm{Ext}^i(Q'(\overline{\mathcal M}_{r'}(S))_{\leq c},V)\to \mathrm{Ext}^i(Q'(\overline{\mathcal M}_{r'}(S))_{\leq r' c},V) \]
  is a pro-isomorphism in $c$, as then the final result follows by passing to a derived limit over $c$, see Lemma~\ref{Ext-Lbar-aux} below.
  This final pro-isomorphism assertion can finally be written out, and it unravels to the statement of Theorem~\ref{first_target}.

  In passing to the derived limit over $c$, we use the following lemma.
\end{proof}

\begin{lemma}
  \label{Ext-Lbar-aux}
  \lean{QprimeFP.short_exact, Lbar.useful_commsq_bicartesian}
  \leanok
  \uses{Lbar, normed_with_aut, BD_eg}
  Assume that in each degree $i$, the map
  \[ (T^{-1})_V - (T^{-1})_{\mathcal M}: \mathrm{Ext}^i(Q'(\overline{\mathcal M}_{r'}(S))_{\leq c},V)\to \mathrm{Ext}^i(Q'(\overline{\mathcal M}_{r'}(S))_{\leq r' c},V) \]
  is a pro-isomorphism in $c$ (i.e., pro-systems of kernels, and of cokernels, are pro-zero). Then
  \[ (T^{-1})_V - (T^{-1})_{\mathcal M}: \mathrm{Ext}^i(Q'(\overline{\mathcal M}_{r'}(S)),V)\to \mathrm{Ext}^i(Q'(\overline{\mathcal M}_{r'}(S)),V).  \]
  is an isomorphism.
\end{lemma}

\begin{proof}
  \uses{first_target}
  \leanok
  We have
  \[ Q'(\overline{\mathcal M}_{r'}(S)) = \bigcup_n Q'(\overline{\mathcal M}_{r'}(S))_{\leq n}, \]
  inducing a resolution
  \[ 0\to \bigoplus_n Q'(\overline{\mathcal M}_{r'}(S))_{\leq n}\to \bigoplus_n Q'(\overline{\mathcal M}_{r'}(S))_{\leq n}\to Q'(\overline{\mathcal M}_{r'}(S))\to 0.  \]
  Passing to a corresponding long exact sequence reduces one to checking that the squares
  \begin{center}
  \begin{tikzcd}
  \prod_n \mathrm{Ext}^i(Q'(\overline{\mathcal M}_{r'}(S))_{\leq n},V)\ar[d]\ar[r] & \prod_n \mathrm{Ext}^i(Q'(\overline{\mathcal M}_{r'}(S))_{\leq n},V)\ar[d]\\
  \prod_n \mathrm{Ext}^i(Q'(\overline{\mathcal M}_{r'}(S))_{\leq n},V)\ar[r] & \prod_n \mathrm{Ext}^i(Q'(\overline{\mathcal M}_{r'}(S))_{\leq n},V)
  \end{tikzcd}
  \end{center}
  are bicartesian (here, horizontal maps are shift minus identity, and vertical maps are $(T^{-1})_V - (T^{-1})_{\mathcal M}$).
  Equivalently, the horizontal maps become isomorphisms on vertical kernels, and vertical cokernels.
  But the vertical kernels and vertical cokernels induce pro-zero systems of abelian groups,
  and then the horizontal kernels and cokernels compute $\varprojlim_n$ and $\varprojlim_n^1$ of their systems, which vanish.
\end{proof}

\begin{proposition}
  \label{Lbar-ses}
  \uses{laurent-measures, Lbar}
  \lean{Lbar.short_exact}
  \leanok
  Let $0 < r < 1$ be a real number, and let $S$ be a profinite set.
  Decomposing $\mathbb Z(\!(T)\!)_r$ into positive and nonpositive coefficients yields a direct sum decomposition
  \[ \mathbb Z(\!(T)\!)_r = T\mathbb Z[\![T]\!]_r \oplus \mathbb Z[T^{-1}]. \]
  This extends to a decomposition of spaces of measures
  \[ \Lm_r(S) = \mathcal L(S,T\mathbb Z(\!(T)\!)_r) = \mathcal L(S,T\mathbb Z[\![T]\!]_r) \oplus \mathcal L(S,\mathbb Z[T^{-1}]) \]
  where $\mathcal M(S,\mathbb Z[T^{-1}]) = \mathbb Z[T^{-1}][S]$ is the free condensed $\mathbb Z[T^{-1}]$-module on $S$.
  Letting $\Lbar_r(S)= \mathcal L(S,T\mathbb Z[\![T]\!]_r)$, we get a short exact sequence of condensed $\mathbb Z[T^{-1}]$-modules
  \[ 0 \to \mathbb Z[T^{-1}][S] \to \Lm_r(S) \to \Lbar_r(S) \to 0. \]
\end{proposition}

\begin{proof}
  \uses{free-png}
  \leanok
  On $\mathbb Z(\!(T)\!)_{r,\leq c}$, only finitely many nonpositive coefficients can possibly be nonzero, and each of them is bounded.
  This shows that the nonpositive summand of $\mathbb Z(\!(T)\!)_r$ is given by $\mathbb Z[T^{-1}]$.
  To pass to profinite $S$, use Proposition~\ref{free-png}.
\end{proof}

\begin{lemma}
  \label{Ext-L}
  \uses{laurent-measures, normed_with_aut}
  \lean{laurent_measures.epi_and_is_iso}
  \leanok
  Let $0 < r < r' < 1$ be real numbers.
  Let $S$ be a profinite set, and let $V$ be an $r$-normed $\mathbb Z[T^{\pm1}]$-module.
  Then $\Ext_{\CondMod_{\mathbb Z[T^{-1}]}}^i(\Lm_{r'}(S), V) = 0$ for all $i > 0$.
  In other words,
  \[ \Ext_{\mathbb Z}^i(\Lm_{r'}(S), V) \xrightarrow{[T⁻¹]_L - [T⁻¹]_V} \Ext_{\mathbb Z}^i(\Lm_{r'}(S), V) \]
  is a bijection for all $i > 0$ and a surjection for $i = 0$.
\end{lemma}

\begin{proof}
  \uses{normed-to-cond-acyclic, Ext-Lbar, Lbar-ses}
  \leanok
  Consider the long exact sequence of Ext-groups
  arising form the short exact sequence (Lemma~\ref{Lbar-ses})
  \[
    0 \to \mathbb Z[T^{-1}][S] \to \Lm_{r'}(S) \to \Lbar_{r'}(S) \to 0
  \]
  by applying $\Ext^*(\_, V)$.

  By Lemma~\ref{acyclic-sheaf} all groups $\Ext_{\Cond(\Ab)}^i(\Z[S], V)$ vanish for $i > 0$.
  And by Lemma~\ref{Ext-Lbar} all groups $\Ext_{\CondMod_{\mathbb Z[T^{-1}]}}^i(\Lbar_{r'}(S), V)$ vanish for $i \ge 0$.
  The result follows.

  The ``In other words'' version can be proved without mentioning $\mathbb Z[T^{-1}]$-linear Ext groups,
  by using the same ingredients and the five lemma.
\end{proof}

\begin{lemma}
  \label{M-ses}
  \lean{laurent_measures.short_exact}
  \leanok
  \uses{laurent-measures, real-measures, theta-exact_finite, CHPNG-Cond-exact, exact-with-constants-lim}
  Let $0 < p' < 1$ be a real number,
  let $S$ be a profinite set,
  and let $r'$ denote $(\tfrac12)^{p'}$.
  There is a short exact sequence of condensed $\mathbb Z[T^{-1}]$-modules
  \[ 0 \to \Lm_{r'}(S) \to \Lm_{r'}(S) \to \Rm(S) \to 0 \]
  where the first map is multiplication by $2T - 1$, and the second is evaluation at $T = \tfrac12$.
\end{lemma}

\begin{proof}
  \uses{theta-exact_finite, CHPNG-Cond-exact, exact-with-constants-lim}
  \leanok
  By Proposition~\ref{CHPNG-Cond-exact} it suffices to show
  that the corresponding sequence of pseudonormed groups is short exact,
  and by Proposition~\ref{exact-with-constants-lim} we may also assume that $S$ is finite. With these reductions, the result is Theorem~\ref{theta-exact_finite}.
  
%  For finite $S$, the sequence is a termwise direct sum (indexed by $s \in S$) of the sequences
%  \[ 0 \to \Lm_{r'}(\{s\}) \to \Lm_{r'}(\{*\}) \to \Rm(\{*\}) \to 0. \]
%  Hence it suffices to show that the sequence
%  \[ 0 \to \Z(\!(T)\!)_{r'} \to \Z(\!(T)\!)_{r'} \to \R \to 0 \]
%  is short exact.
%  The Lemmas~\ref{theta.surjective} and~\ref{theta.continuous} show that the evaluation map is indeed
%  continuous and surjective.
%  A computation (omitted here for now, but done in Lean) shows that the kernel of the evaluation map
%  is exactly $(2T-1) \Z(\!(T)\!)_{r'}$.
\end{proof}

\begin{theorem}[Clausen--Scholze]
  \label{main-thm}
  \uses{p-banach, real-measures}
  \lean{liquid_tensor_experiment}
  \leanok
  Let $0 < p' < p \le 1$ be real numbers,
  let $S$ be a profinite set,
  and let $V$ be a $p$-Banach space.
  Let $\Rm(S)$ be the space of real $p'$-measures on $S$. Then
  \[
\Ext^i_{\Cond(\Ab)}(\Rm(S),V) = 0
\]
  for $i \ge 1$.
\end{theorem}

\begin{proof}
  \uses{M-ses, Ext-L, p-banach-r-normed}
  \leanok
  Recall from Lemma \ref{M-ses} the short exact sequence
  \[ 0 \to \Lm_{r'}(S) \to \Lm_{r'}(S) \to \Rm(S) \to 0. \]
  Apply to this $\text{Ext}^*(\_, V)$ to obtain a long exact sequence.
  Note that $T$ acts on $V$ via multiplication by $\tfrac12$ (by Lemma~\ref{p-banach-r-normed}.
  Hence we can use Lemma~\ref{Ext-L} to obtain isomorphisms between the Ext-groups involving $\Lm_{r'}(S)$, for $i > 0$,
  and a surjection for $i = 0$.
  The result follows.
\end{proof}

% vim: ts=2 et sw=2 sts=2
