\subsection{Variants of normed groups}
\label{sec:normed_groups}

Normed groups are well-studied objects.
In this text it will be helpful to work with
the more general notion of \emph{semi-normed group}.
This drops the separation axiom $\|x\| = 0 \iff x = 0$
but is otherwise the same as a normed group.

The main difference is that this includes ``uglier'' objects,
but creates a ``nicer'' category:
semi-normed groups need not be Hausdorff,
but quotients by arbitrary (possibly non-closed) subgroups
are naturally semi-normed groups.

Nevertheless, there is the occasional use for the more restrictive
notion of normed group, when we come to polyhedral lattices below
(see Section~\ref{sec:polyhedral_lattice}).

In this text, a morphism of (semi)-normed groups will always be bouned.
If the morphism is supposed to be norm-nonincreasing,
this will be mentioned explicitly.

\begin{definition}
  \label{normed_with_aut}\alsoIn{cha:two}
  \lean{normed_with_aut}
  \leanok
  Let $r > 0$ be a real number.
  An \emph{$r$-normed $\mathbb Z[T^{\pm 1}]$-module}
  is a semi-normed group $V$
  endowed with an automorphism $T \colon V \to V$ such that
  for all $v \in V$ we have $\|T(v)\| = r\|v\|$.
\end{definition}

The remainder of this subsection sets up some algebraic variants of semi-normed groups.

\begin{definition}
  \label{pseudo_normed_group}\alsoIn{cha:two}
  \lean{pseudo_normed_group}
  \leanok
  A \emph{pseudo-normed group} is an abelian group $(M,+)$,
  together with an increasing filtration $M_c \subseteq M$ of subsets $M_c$ indexed by $\mathbb R_{\ge 0}$,
  such that each $M_c$ contains $0$, is closed under negation,
  and $M_{c_1} + M_{c_2} \subseteq M_{c_1 + c_2}$. An example would be $M=\mathbb{R}$ or $M=\mathbb{Q}_p$ with $M_c :=\{x\,:\,|x|\leq c\}$.

  A pseudo-normed group $M$ is \emph{exhaustive} if $\varinjlim_c M_c = M$.
\end{definition}

All pseudo-normed groups that we consider will have a topology on the filtration sets $M_c$.
The most general variant is the following notion.

\begin{definition}
  \label{chpng}\alsoIn{cha:two}
  \uses{pseudo_normed_group}
  \lean{comphaus_filtered_pseudo_normed_group, profinitely_filtered_pseudo_normed_group}
  \leanok
  A pseudo-normed group~$M$ is \emph{CH-filtered}
  if each of the sets $M_c$ is endowed with a topological space structure
  making it a compact Hausdorff space,
  such that following maps are all continuous:
  \begin{itemize}
    \item the inclusion $M_{c_1} \to M_{c_2}$ (for $c_1 \le c_2$);
    \item the negation $M_c \to M_c$;
    \item the addition $M_{c_1} \times M_{c_2} \to M_{c_1 + c_2}$.
  \end{itemize}

  The pseudo-normed group~$M$ is \emph{profinitely filtered} if moreover
  the filtration sets~$M_c$ are totally disconnected, making them profinite sets.
\end{definition}

\begin{remark}
  The topologies on the filtration sets $M_c$ will induce a topology on $M$: the colimit topology.
  If $M$ is some sort of normed group, then this topology is typically genuinely different from the norm topology.
\end{remark}

\begin{definition}
  \label{chpng-hom}\alsoIn{cha:two}
  \uses{chpng}
  \lean{comphaus_filtered_pseudo_normed_group_hom}
  \leanok
  A \emph{morphism} of CH-filtered pseudo-normed groups $M \to N$
  is a group homomorphism $f \colon M \to N$ that is
  \begin{itemize}
    \item \emph{bounded}:
      there is a constant $C$
      such that $x \in M_c$ implies $f(x) \in N_{Cc}$;
    \item \emph{continuous}:
      for one (or equivalently all) constants $C$ as above,
      the induced map $M_c \to N_{Cc}$ is
      a morphism of profinite sets, i.e. continuous.
  \end{itemize}
  The reason the two definitions of continuity are equivalent is that a continuous injection from a compact space to a Hausdorff space must be a topological embedding.

  A morphism $f \colon M \to N$ is \emph{strict} if $x \in M_c$ implies $f(x) \in N_c$
  (in other words, if we can take $C = 1$ in the boundedness condition above).
\end{definition}

We will also consider the analogue of an $r$-normed $\mathbb Z[T^{-1}]$-module in the pseudo-normed setting.

\begin{definition}
  \label{chpng-Tinv}\alsoIn{cha:two}
  \uses{chpng-hom}
  \lean{comphaus_filtered_pseudo_normed_group_with_Tinv, comphaus_filtered_pseudo_normed_group_with_Tinv_hom}
  \leanok
  Let $r'$ be a positive real number.
  A CH-filtered pseudo-normed group $M$
  has an \emph{$r'$-action of $T^{-1}$}
  if it comes endowed with a distinguished morphism of CH-filtered pseudo-normed groups
  $T^{-1} \colon M \to M$ that is bounded by $r'^{-1}$:
  if $x \in M_c$ then $T^{-1}x \in M_{c/r'}$.

  A morphism of CH-filtered pseudo-normed groups with $r'$-action of $T^{-1}$
  is a morphism $f \colon M \to N$ of CH-filtered pseudo-normed groups that commutes with the action of~$T^{-1}$.
\end{definition}

% vim: ts=2 et sw=2 sts=2
